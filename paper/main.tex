\documentclass[conference]{IEEEtran}
\IEEEoverridecommandlockouts
% The preceding line is only needed to identify funding in the first footnote. If that is unneeded, please comment it out.
\usepackage{cite}
\usepackage{amsmath,amssymb,amsfonts}
\usepackage{algorithmic}
\usepackage{graphicx}
\usepackage{textcomp}
\usepackage{xcolor}
\usepackage[spanish]{babel}
\usepackage{url}
\urlstyle{rm}

\usepackage{xspace}
\usepackage{color}

%Some comments
\newcommand\todo[1]{\mynote{TODO}{#1}}
\newcommand\wtf[1]{\mynote{WTF}{#1????}}
\newcommand\fix[1]{\mynote{FIX}{#1}}
\newcommand\reform[1]{\mynote{reform}{#1}} %reformulate
\newcommand{\here}[0]{\bigskip \mbox{\bf**************HERE*************}}

%Personal comments
\newcommand\pl[1]{\mynote{PL}{#1 }}
\newcommand\et[1]{\mynote{ET}{#1}}
\newcommand\hf[1]{\mynote{HF}{#1}}

%Connectors
\newcommand{\ie}[0]{{\em i.e.,~}}
\newcommand{\eg}[0]{{\em e.g.,~}}
\newcommand{\etal}[0]{{\em et al.}}
\newcommand{\aka}[0]{{\em a.k.a.}}
%\newcommand{\ala}[0]{{\em al a}}

%Languages
\newcommand{\javascript}[0]{\mbox{JavaScript}\xspace}
\newcommand{\deloreanjs}[0]{\mbox{DeloreanJS}\xspace}
\newcommand{\actionscript}[0]{\mbox{ActionScript}\xspace}
\newcommand{\java}[0]{\mbox{Java}\xspace}

%Aspect languages
\newcommand{\aspectscript}[0]{\mbox{AspectScript}\xspace}
\newcommand{\aspectj}[0]{\mbox{AspectJ}\xspace}


%Web
\newcommand{\async}[0]{\mbox{SyncAS}\xspace}
\newcommand{\promise}[0]{\mbox{Promise}\xspace}
\newcommand{\asyncjs}[0]{\mbox{Async.Js}\xspace}
\newcommand{\ajax}[0]{\mbox{Ajax}\xspace}
\newcommand{\synccc}[0]{\mbox{Sync/cc}\xspace}
\newcommand{\callcc}[0]{\mbox{continuations}\xspace}
\newcommand{\jquery}[0]{\mbox{jQuery}\xspace}
\newcommand{\rhino}[0]{\mbox{Rhino}\xspace}
\newcommand{\ecmascript}[0]{\mbox{ECMAScript}\xspace}
\newcommand{\browserify}[0]{\mbox{Browserify}\xspace}
\newcommand{\unwinder}[0]{\mbox{Unwinder}\xspace}
\newcommand{\nodejs}[0]{\mbox{Nodejs}\xspace}
\newcommand{\tongoy}[0]{\mbox{Tongoy-UCN}\xspace}


%==================================================
%Helper commands
\definecolor{red}{rgb}{1,0,0}
\newcommand\change[1]{\textcolor{red}{#1}}
\newcommand{\kw}[1]{{\bf#1}}
\newcommand{\parhead}[1]{\noindent{\bf {\em #1.}}}

\newcommand{\mynote}[2]{
\fbox{\bfseries\sffamily\scriptsize#1}
  {\small\textsf{\emph{#2}}}
\fbox{\bfseries\sffamily\scriptsize }
}

\newcommand{\super}[1]{\ensuremath{^{\textrm{#1}}}}
\newcommand{\sub}[1]{\ensuremath{_{\textrm{#1}}}}

\usepackage{graphicx}
\usepackage{xspace}
%\usepackage{url}
\usepackage{listings}
% The following packages are used in the Scheme definition


\lstdefinelanguage{java}{
morekeywords={%Java keywords
float, public, interface, class, static, void, extends, implements, final,
boolean, return, new, abstract, super, for, int, package, private, import,
protected, this, throw, try, finally, if, else, while, instance of, &&, ||},
	sensitive=true,
	morecomment=[l]{//},
	morecomment=[s]{/*}{*/},
	morestring=[b]",
}

\lstdefinelanguage{tracematches}
{morekeywords={%Java keywords
float, public, interface, class, static, void, extends, implements, final,
boolean, return, new, abstract, super, for, int, package, private, import,
protected, this, throw, try, finally, if, else, while, instance of, &&, ||,
%Aspectj keywords
before, around, after, returning, tracematch, sym, call, exec},
	sensitive=true,
	morecomment=[l]{//},
	morecomment=[s]{/*}{*/},
	morestring=[b]",
}


\lstdefinelanguage{JavaScript}
	{morekeywords={%Java keywords
	               true, false, return, new, function, var,
	               this, throw, try, finally, if, else, while, for, instanceof, 
	               %Others
	               async, await},
	sensitive=true,
	morecomment=[l]{//},
	morecomment=[s]{/*}{*/},
	morestring=[b]",
}

\lstset{
	frame=,
	language=JavaScript,
	xleftmargin=6pt,
	stepnumber=1,
	numbers=none,
	numbersep=5pt,
	numberstyle=\ttfamily\tiny,
	belowcaptionskip=\bigskipamount,
	captionpos=b,
	escapeinside={*'}{'*},
	tabsize=2,
	emphstyle={\bf},
	stringstyle=\mdseries\rmfamily,
	showspaces=false,
	keywordstyle=\bfseries,
    basewidth=0.5em,
	columns=fixed,
	%basicstyle=\footnotesize\sffamily,
         basicstyle=\scriptsize,
	showstringspaces=false,
	morecomment=[l]\%,
	commentstyle=\it
}

%\newcommand{\co}[1]{\mbox{{\small #1}\xspace}}
\newcommand{\co}[1]{\mbox{{\fontfamily{phv}\selectfont\scriptsize #1}\xspace}}
\newcommand{\env}[1]{\framebox{{\fontfamily{phv}\selectfont\scriptsize
#1}\xspace}}
\newcommand{\lambdac}[0]{$\lambda$~}
%\renewcommand{\ttdefault}{cmtt}
%\renewcommand{\ttdefault}{txtt}


\def\BibTeX{{\rm B\kern-.05em{\sc i\kern-.025em b}\kern-.08em
    T\kern-.1667em\lower.7ex\hbox{E}\kern-.125emX}}
\begin{document}

\title{DeloreanJS: Un Debugger en el Tiempo para JavaScript\\
}

\author{\IEEEauthorblockN{Paul Leger}
\IEEEauthorblockA{\textit{Escuela de Ingenier\'ia} \\
\textit{Universidad Cat\'olica del Norte}\\
Coquimbo, Chile \\
pleger@ucn.cl}
\and
\IEEEauthorblockN{AAAA BBBB}
\IEEEauthorblockA{\textit{Escuela de Ingenier\'ia} \\
\textit{Universidad Cat\'olica del Norte}\\
Coquimbo, Chile \\
aaaa.bbbb@alumnos.ucn.cl}
\and
\IEEEauthorblockN{XXXX YYYY}
\IEEEauthorblockA{\textit{Escuela de Ingenier\'ia} \\
\textit{Universidad Cat\'olica del Norte}\\
Coquimbo, Chile \\
xxxx.yyyy@alumnos.ucn.cl}
}

\maketitle

\begin{abstract}
Aplicaciones Web, usando \javascript, son desarrolladas cada vez con mayor frecuencia. Como en la mayor\'ia de los entornos de desarrollos, una aplicaci\'on Web puede adquirir defectos de software (conocido como {\em bugs}) cuyos s\'intomas se aprecian durante este desarrollo e incluso, siendo peor, en su puesta en producci\'on. Por ello, el uso de debuggers son sumamente \'utiles para detectar estos bugs. Lamentablemente, los actuales debuggers solamente avanzan hacia adelante en la ejecuci\'on para detectar el bug y no permiten retornar hacia un punto anterior para tomar acciones asociadas al bug detectado. Por ejemplo, probar si el mismo bug podr\'ia gatillarse con otro valor de una variable. Usando el concepto de continuaciones, este art\'iculo presenta un debugger para \javascript que permite devolverse en el tiempo con el fin que un programador pueda volver a verificar y probar el contexto alrededor de un bug. Este debugger, llamado \deloreanjs, ha sido implementado como una extensi\'on para Mozilla Firefox con el fin que la comunidad de desarrolladores puedan usarlo.         
\end{abstract}

\begin{IEEEkeywords}
\deloreanjs, \javascript, Debugger, Web application, Continuations
\end{IEEEkeywords}

\section{Introducci\'on}
\label{sec:intro}

La industria del software se mueve fuertemente hacia el desarrollo de aplicaciones Web, siendo testigo un gran n\'umero de migraciones de aplicaciones {\em standalone} a la Web; los ejemplos van desde convertir un documento Word a PDF~\cite{smallpdf} hasta un sistema ERP ({\em Enterprise Resource Planning})~\cite{erpOrcale}. Para construir estas aplicaciones, uno de los lenguajes m\'as usados en \javascript, cuya presencia en el entorno de la Web es alrededor de 95\%~\cite{jsuses}. Por ello, cada vez estas aplicaciones se han vuelto más complejas y con mayor riesgo de introducir defectos de software (\aka {\em bugs}). 

Detectar y reparar bugs representa una de las tareas m\'as costosa en el proceso de desarrollo de software, y el mundo de la Web no es la excepci\'on. Para ello, un conjunto de debuggers se han propuestos, desde el que incluye un navegador (\eg Mozilla Firefox developer tools) hasta avanzados como PECCit~\cite{azar:2016} y JARDIS~\cite{barrAl:fse2016}. Lamentablemente, la gran parte de estos debuggers solo permiten mostrar la ocurrencia de un bug sin entregar la posibilidad de retroceder {\em en el tiempo} para tal vez repararlo mientras se ejecuta el programa o manipular el estado de un programa alrededor del bug con el fin de mejorar su comprensi\'on.       



\bigskip
\bigskip
\bigskip

\pl{La Web con aplicaciones JavaScript la llevan}\\

\pl{Siempre hay errores, con mayor raz\'on en aplicaciones grandes. Los programadores generalmente encuentran sintomas de un error que se ocurrio mucho antes}\\

\pl{Debuggear es una tarea sumamente compleja para encontrar esos errores (frase tipica de debugger), existe muchos tipos de debuggers para \javascript (dar citas)}\\

\pl{Lamentablemente los debuggers actuales son solamente analsis post-morter, donde si quiere entender claramente el contexto de un error se debe ejecutar el mismo programa varias veces y con el mismo contexto de ejecución. Además, como son debuggers post-morter significa que la aplicacion debe parar su ejecucion para entender el bug}\\

\pl{Nosotros proponemos un debugger en el tiempo, llamado \deloreanjs, que no requiere finalizar la ejecución de la aplicacion. Con ello, se puede volver al momento donde ocurrio el bug y cambiar valores para entender mas claramente bug}\\

\bigskip

El art\'iculo est\'a organizado como sigue. La secci\'on~\ref{sec:tour} presenta diferentes ejemplos de aplicaci\'on de \deloreanjs, donde se puede apreciar la interfaz gr\'afica de nuestra propuesta. Luego, se presenta \deloreanjs, detallando c\'omo se usa el concepto de continuaciones. En la secci\'on~\ref{sec:rw} se discuten herramientas similares. Finalmente, secci\'on~\ref{sec:conc} concluye y describe lineamientos sobre el trabajo futuro de nuestra propuesta.      

\smallskip

{\bf {\em Disponibilidad.}} La implementaci\'on de \deloreanjs se encuentra en {\tt \url{http://github.com/fruizrob/delorean}} y su extensi\'on en Mozilla Firefox en {\tt \url{http://xxx.com}}. 



\section{Un Tour por \deloreanjs}
\label{sec:tour}

\pl{Cada ejemplo debe tener el nombre de lo que hace}

\subsection{Ejemplo 1}
\label{sec:tour1}

\pl{Explicar con imagenes el ejemplo 1}

\subsection{Ejemplo 2}
\label{sec:tour2}

\pl{Explicar con imagenes el ejemplo 2}

\subsection{Ejemplo 3}
\label{sec:tour3}

\pl{Explicar con imagenes el ejemplo 3}

\bigskip

\section{\deloreanjs}
\label{sec:deloreanjs}


\subsection{Continuaciones}
\label{sec:continuaciones}

\pl{Explicar continuaciones~\cite{fw84}}

\subsection{Capturando el Estado}
\label{sec:estados}

\pl{Explicar su forma de capturar estados}

\subsection{Extensi\'on en el Navegador}

\pl{Explicar la aplicaci\'on en Mozilla Firefox}

\bigskip

\section{Trabajo Relacionado}
\label{sec:rw}

\pl{Aqui se debe comparar con tres herramientas similares a \deloreanjs}

\bigskip

\parhead{Herramienta 1}
\smallskip

\parhead{Herramienta 2}
\smallskip

\parhead{Herramienta 3}
\smallskip

\bigskip

\section{Conclusiones y Trabajo Futuro}
\label{sec:conc}

\pl{TODO}

\smallskip

\parhead{Trabajo Futuro}

\bibliographystyle{IEEEtran}
\bibliography{djs}


\end{document}
